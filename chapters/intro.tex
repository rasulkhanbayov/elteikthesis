\chapter{Introduction} % Introduction
\label{ch:intro}

\section{Motivation}
We are living in an increasingly hyperconnected world and it is not just people that are connected anymore. The Internet of Things (IoT) is creating new connectivity requirements and scenarios for our world. This connectivity is provided by Communications Service Providers (CSP) which currently facing an exponential rise in the number of customers in their network. This leads to capital expenditures, operating expenses as well as inability to offer new services and applications for users dynamically. IT organizations use the deployment of shared and virtual infrastructure to manage the delivery of multiple networked applications and expand resource utilization in a more efficient way.

Network Functions Virtualization (NFV) applies virtualization technologies to the long-established network and service functions, and associated applications that CSPs use to reduce the cost and improve time-to-market in their network infrastructure~\cite{nfv-for-dummies}. Cloud technologies, especially Ericsson Orchestrator Cloud Manager offers services such as building cloud infrastructure for virtualized applications, secure GUI, and open API for all management services. Cloud Manager enables orchestration of resources across different Virtualized Infrastructure Managers (VIM) (OpenStack or VMware) additionally Kubernetes to 5G and IoT customers.

The main purpose of this thesis is to build an integrated platform where users are able to manage the life cycle of NFV and IT workloads. By taking advantage of this platform, they do not need to be using various applications to create packages, upload YAML format files and deploy/delete resources from Kubernetes.

\section{Thesis structure}
Chapter 2 User Documentation contains:
\begin{itemize}
  \item A short description of the cloud and cloud computing. See~\autoref{sec:cloud}.
  \item A brief introduction to the tools used in the front-end and back-end developments, all the requirements for the installation steps, and running the project. See~\autoref{sec:install-guide}.
  \item A detailed overview of the usage of the application from the client's perspective. See~\autoref{sec:usage-system}.
\end{itemize}

Chapter 3 Developer Documentation contains:
\begin{itemize}
  \item A detailed specification of the application. See~\autoref{sec:specification}.
  \item A comprehensive description for the REST architectural style.  See~\autoref{subsec:api-overview}.
  \item Explanation to the communication of Spring Boot with React Js and PostgreSQL database. See~\autoref{subsec:reactjs-spring} and~\autoref{subsec:interaction-db}.
  \item Advantages of Java Kubernetes client library and its utilization in the application.  See~\autoref{subsec:kubernetes-spring-2}.
  \item Usage of Jasypt java library for basic encryption and Spring boot authentication. Check~\autoref{subsec:jasypt} and~\autoref{subsec:basic-auth} respectively.
  \item Testing, a short description of Spring MVC test framework, test templates for different HTTP requests, and results of integration tests. See~\autoref{sec:testing}.
\end{itemize}
